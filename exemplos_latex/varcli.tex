% compilar este documento com o comando
% pdflatex '\providecommand{\sendtoprinter}{VALOR}% compilar este documento com o comando
% pdflatex '\providecommand{\sendtoprinter}{VALOR}% compilar este documento com o comando
% pdflatex '\providecommand{\sendtoprinter}{VALOR}% compilar este documento com o comando
% pdflatex '\providecommand{\sendtoprinter}{VALOR}\input{varcli.tex}'
% em que VALOR é true ou false, dependendo da opção desejada.
\documentclass[a4,12pt]{article}
\usepackage[utf8]{inputenc}
\usepackage[portuguese]{babel}
\usepackage{graphicx}
\usepackage{ifthen}
% valor padrão
\providecommand\sendtoprinter{false}

\title{Testando a linha de comando no \LaTeX}
\author{Melissa}

\begin{document}
\maketitle

Neste documento, estamos testando o que acontece quando compilamos nosso arquivo \verb+.tex+ usando argumentos extras na linha de comando.

Vamos mostrar que a Figura~\ref{fig:penelope} só aparece na versão digital. Na versão impressa, temos apenas o \emph{float} correpondente à figura (com referências corretas).

\begin{figure}[!h]
   \begin{center}
      \ifthenelse{\equal{\sendtoprinter}{true}}
      {
        \framebox[5cm]{ 
          \raisebox{0pt}[3cm]
          {
            \parbox{4cm}
            {
              {\footnotesize{Esta é uma figura que só aparecerá na versão impressa.}}
            } % Fim parbox
          } % Fim raisebox
        } % Fim framebox
      } % Fim If
      { % Else
        \includegraphics[width=5cm]{penelope.jpg}
      } % Fim Else
      \caption{Foto da minha gata Penélope.\label{fig:penelope}}
   \end{center}
\end{figure}
   
\end{document}



'
% em que VALOR é true ou false, dependendo da opção desejada.
\documentclass[a4,12pt]{article}
\usepackage[utf8]{inputenc}
\usepackage[portuguese]{babel}
\usepackage{graphicx}
\usepackage{ifthen}
% valor padrão
\providecommand\sendtoprinter{false}

\title{Testando a linha de comando no \LaTeX}
\author{Melissa}

\begin{document}
\maketitle

Neste documento, estamos testando o que acontece quando compilamos nosso arquivo \verb+.tex+ usando argumentos extras na linha de comando.

Vamos mostrar que a Figura~\ref{fig:penelope} só aparece na versão digital. Na versão impressa, temos apenas o \emph{float} correpondente à figura (com referências corretas).

\begin{figure}[!h]
   \begin{center}
      \ifthenelse{\equal{\sendtoprinter}{true}}
      {
        \framebox[5cm]{ 
          \raisebox{0pt}[3cm]
          {
            \parbox{4cm}
            {
              {\footnotesize{Esta é uma figura que só aparecerá na versão impressa.}}
            } % Fim parbox
          } % Fim raisebox
        } % Fim framebox
      } % Fim If
      { % Else
        \includegraphics[width=5cm]{penelope.jpg}
      } % Fim Else
      \caption{Foto da minha gata Penélope.\label{fig:penelope}}
   \end{center}
\end{figure}
   
\end{document}



'
% em que VALOR é true ou false, dependendo da opção desejada.
\documentclass[a4,12pt]{article}
\usepackage[utf8]{inputenc}
\usepackage[portuguese]{babel}
\usepackage{graphicx}
\usepackage{ifthen}
% valor padrão
\providecommand\sendtoprinter{false}

\title{Testando a linha de comando no \LaTeX}
\author{Melissa}

\begin{document}
\maketitle

Neste documento, estamos testando o que acontece quando compilamos nosso arquivo \verb+.tex+ usando argumentos extras na linha de comando.

Vamos mostrar que a Figura~\ref{fig:penelope} só aparece na versão digital. Na versão impressa, temos apenas o \emph{float} correpondente à figura (com referências corretas).

\begin{figure}[!h]
   \begin{center}
      \ifthenelse{\equal{\sendtoprinter}{true}}
      {
        \framebox[5cm]{ 
          \raisebox{0pt}[3cm]
          {
            \parbox{4cm}
            {
              {\footnotesize{Esta é uma figura que só aparecerá na versão impressa.}}
            } % Fim parbox
          } % Fim raisebox
        } % Fim framebox
      } % Fim If
      { % Else
        \includegraphics[width=5cm]{penelope.jpg}
      } % Fim Else
      \caption{Foto da minha gata Penélope.\label{fig:penelope}}
   \end{center}
\end{figure}
   
\end{document}



'
% em que VALOR é true ou false, dependendo da opção desejada.
\documentclass[a4,12pt]{article}
\usepackage[utf8]{inputenc}
\usepackage[portuguese]{babel}
\usepackage{graphicx}
\usepackage{ifthen}
% valor padrão
\providecommand\sendtoprinter{false}

\title{Testando a linha de comando no \LaTeX}
\author{Melissa}

\begin{document}
\maketitle

Neste documento, estamos testando o que acontece quando compilamos nosso arquivo \verb+.tex+ usando argumentos extras na linha de comando.

Vamos mostrar que a Figura~\ref{fig:penelope} só aparece na versão digital. Na versão impressa, temos apenas o \emph{float} correpondente à figura (com referências corretas).

\begin{figure}[!h]
   \begin{center}
      \ifthenelse{\equal{\sendtoprinter}{true}}
      {
        \framebox[5cm]{ 
          \raisebox{0pt}[3cm]
          {
            \parbox{4cm}
            {
              {\footnotesize{Esta é uma figura que só aparecerá na versão impressa.}}
            } % Fim parbox
          } % Fim raisebox
        } % Fim framebox
      } % Fim If
      { % Else
        \includegraphics[width=5cm]{penelope.jpg}
      } % Fim Else
      \caption{Foto da minha gata Penélope.\label{fig:penelope}}
   \end{center}
\end{figure}
   
\end{document}



